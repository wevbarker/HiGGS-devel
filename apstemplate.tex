\documentclass[aps,prd,reprint,preprintnumbers,superscriptaddress,showpacs,floatfix]{revtex4-2}
\usepackage[utf8]{inputenc}
\usepackage{parskip}
\usepackage{amssymb}
\usepackage{hhline}
\usepackage{amsmath}
\usepackage{bm}
\usepackage[dvipsnames]{xcolor}
\usepackage{xspace}
\usepackage{multirow,tabularx}
\usepackage{siunitx}
\usepackage{textgreek}
\usepackage{upgreek}
\usepackage{multirow}
\usepackage{graphicx}
\usepackage{pifont}
\usepackage{xstring}
\usepackage{etoolbox}
\usepackage{stmaryrd}
\usepackage{notoccite}
\usepackage{natbib}
\usepackage{lineno}
\usepackage{tensor}
\usepackage[thinlines,thicklines]{easybmat}
\usepackage{bbm}
\usepackage{accents}

\DeclareSIUnit\parsec{pc} %define a parsec unit
\DeclareSIUnit\littleh{\mathsf{h}} %define normalised Hubble number
\DeclareSIUnit\ccs{{m_{\text{p}}}^2} %use of reduced Planck mass as unit
\DeclareSIUnit\nothing{\relax} %enable use of dimensionless quantities via siunitx package



% Journal names 
\def\aj{\rm{AJ}}
\def\actaa{\rm{Acta Astron.}}
\def\araa{\rm{ARA\&A}}
\def\apj{\rm{ApJ}}
\def\apjl{\rm{ApJ}}
\def\apjs{\rm{ApJS}}
\def\ao{\rm{Appl.~Opt.}}
\def\apss{\rm{Ap\&SS}}
\def\aap{\rm{A\&A}}
\def\aapr{\rm{A\&A~Rev.}}
\def\aaps{\rm{A\&AS}}
\def\azh{\rm{AZh}}
\def\baas{\rm{BAAS}}
\def\bac{\rm{Bull. astr. Inst. Czechosl.}}
\def\caa{\rm{Chinese Astron. Astrophys.}}
\def\cjaa{\rm{Chinese J. Astron. Astrophys.}}
\def\icarus{\rm{Icarus}}
\def\jcap{\rm{J. Cosmology Astropart. Phys.}}
\def\jrasc{\rm{JRASC}}
\def\memras{\rm{MmRAS}}
\def\mnras{\rm{MNRAS}}
\def\na{\rm{New A}}
\def\nar{\rm{New A Rev.}}
\def\pra{\rm{Phys.~Rev.~A}}
\def\prb{\rm{Phys.~Rev.~B}}
\def\prc{\rm{Phys.~Rev.~C}}
\def\prd{\rm{Phys.~Rev.~D}}
\def\pre{\rm{Phys.~Rev.~E}}
\def\prl{\rm{Phys.~Rev.~Lett.}}
\def\pasa{\rm{PASA}}
\def\pasp{\rm{PASP}}
\def\pasj{\rm{PASJ}}
\def\rmxaa{\rm{Rev. Mexicana Astron. Astrofis.}}
\def\qjras{\rm{QJRAS}}
\def\skytel{\rm{S\&T}}
\def\solphys{\rm{Sol.~Phys.}}
\def\sovast{\rm{Soviet~Ast.}}
\def\ssr{\rm{Space~Sci.~Rev.}}
\def\zap{\rm{ZAp}}
\def\nat{\rm{Nature}}
\def\iaucirc{\rm{IAU~Circ.}}
\def\aplett{\rm{Astrophys.~Lett.}}
\def\apspr{\rm{Astrophys.~Space~Phys.~Res.}}
\def\bain{\rm{Bull.~Astron.~Inst.~Netherlands}}
\def\fcp{\rm{Fund.~Cosmic~Phys.}}
\def\gca{\rm{Geochim.~Cosmochim.~Acta}}
\def\grl{\rm{Geophys.~Res.~Lett.}}
\def\jcp{\rm{J.~Chem.~Phys.}}
\def\jgr{\rm{J.~Geophys.~Res.}}
\def\jqsrt{\rm{J.~Quant.~Spec.~Radiat.~Transf.}}
\def\memsai{\rm{Mem.~Soc.~Astron.~Italiana}}
\def\nphysa{\rm{Nucl.~Phys.~A}}
\def\physrep{\rm{Phys.~Rep.}}
\def\physscr{\rm{Phys.~Scr}}
\def\planss{\rm{Planet.~Space~Sci.}}
\def\procspie{\rm{Proc.~SPIE}}

\parskip 1mm
\parindent 2mm


\newcommand{\aneq}{\mathrel{\smash[t]{\stackrel{{\text{an}}}{=}}}} %analogue equality

\newcommand{\comment}[1]{\par {\sffamily  \color{red} #1 \par}} %for comments during editing

\newrobustcmd{\stall}[1]{%
\IfEqCase{#1}{%
  {1}{State I\xspace}%
  {2}{State II\xspace}%
}[\packageError{cosmicclass}{Unidentified Critical Case: #1}{}]%
}
\newrobustcmd{\numbercosmicclass}{14}%
\newrobustcmd{\cosmicclass}[1]{%
\IfEqCase{#1}{%
{1}{Class~\textsuperscript{3}E\xspace}%
{2}{Class~\textsuperscript{2}A\xspace}%
{8}{Class~\textsuperscript{4}H\xspace}%
{9}{Class~\textsuperscript{5}M\xspace}%
{10}{Class~\textsuperscript{5}M\xspace}%
{11}{Class~\textsuperscript{4}I\xspace}%
{12}{Class~\textsuperscript{4}J\xspace}%
{14}{Class~\textsuperscript{4}H\xspace}%
{13}{Class~\textsuperscript{4}K\xspace}%
{16}{Class~\textsuperscript{3}C\xspace}%
{15}{Class~\textsuperscript{3}D\xspace}%
{21}{Class~\textsuperscript{3}F\xspace}%
{20}{Class~\textsuperscript{3}F\xspace}%
{23}{Class~\textsuperscript{4}L\xspace}%
{22}{Class~\textsuperscript{3}F\xspace}%
{26}{Class~\textsuperscript{4}N\xspace}%
{27}{Class~\textsuperscript{3}E\xspace}%
{24}{Class~\textsuperscript{3}F\xspace}%
{25}{Class~\textsuperscript{3}F\xspace}%
{30}{Class~\textsuperscript{3}E\xspace}%
{31}{Class~\textsuperscript{3}G\xspace}%
{28}{Class~\textsuperscript{4}N\xspace}%
{29}{Class~\textsuperscript{2}A\xspace}%
{35}{Class~\textsuperscript{3}E\xspace}%
{34}{Class~\textsuperscript{3}G\xspace}%
{33}{Class~\textsuperscript{4}N\xspace}%
{32}{Class~\textsuperscript{4}N\xspace}%
{39}{Class~\textsuperscript{3}G\xspace}%
{38}{Class~\textsuperscript{3}G\xspace}%
{37}{Class~\textsuperscript{2}A\xspace}%
{36}{Class~\textsuperscript{2}B\xspace}%
{40}{Class~\textsuperscript{2}B\xspace}%
{41}{Class~\textsuperscript{2}A\xspace}%
{null}{Class~\textsuperscript{3}C*\xspace}%
{cnull}{Class~\textsuperscript{2}A*\xspace}%
}[\packageError{cosmicclass}{Unidentified Critical Case: #1}{}]%
}
\newrobustcmd{\criticalcase}[1]{%
\IfEqCase{#1}{%
{1}{Case 1\xspace}%
{2}{Case 2\xspace}%
{8}{Case 8\xspace}%
{9}{Case \textsuperscript{*1}9\xspace}%
{10}{Case \textsuperscript{*3}10\xspace}%
{11}{Case \textsuperscript{*4}11\xspace}%
{12}{Case 12\xspace}%
{14}{Case 14\xspace}%
{13}{Case \textsuperscript{*2}13\xspace}%
{16}{Case 16\xspace}%
{15}{Case 15\xspace}%
{21}{Case 21\xspace}%
{20}{Case 20\xspace}%
{23}{Case 23\xspace}%
{22}{Case 22\xspace}%
{26}{Case \textsuperscript{*6}26\xspace}%
{27}{Case 27\xspace}%
{24}{Case 24\xspace}%
{25}{Case \textsuperscript{*5}25\xspace}%
{30}{Case \textsuperscript{*7}30\xspace}%
{31}{Case \textsuperscript{*8}31\xspace}%
{28}{Case 28\xspace}%
{29}{Case 29\xspace}%
{35}{Case \textsuperscript{*9}35\xspace}%
{34}{Case 34\xspace}%
{33}{Case 33\xspace}%
{32}{Case 32\xspace}%
{39}{Case 39\xspace}%
{38}{Case 38\xspace}%
{37}{Case 37\xspace}%
{36}{Case \textsuperscript{*10}36\xspace}%
{40}{Case 40\xspace}%
{41}{Case 41\xspace}%
}[\packageError{criticalcase}{Unidentified Critical Case: #1}{}]%
}
 %this formats the Class X notation for the various cosmologies
\newrobustcmd{\indiq}[2][placeholder]{%
\IfEqCase{#1}{%
  {placeholder}{%
    \IfEqCase{#2}{%
    {1}{k}%
    {2}{kl}%
    {3}{klo}%
    }%
  }%
}[#1]%
}%

\newrobustcmd{\pic}[2][placeholder]{%
\IfEqCase{#2}{%
{B0p}{\varphi}%
{B1p}{\tensor{\overset{\wedge}{\varphi}}{_{\overline{\indiq[#1]{2}}}}}%
{B1m}{\tensor{\varphi}{_{\perp\overline{\indiq[#1]{1}}}}}%
{B2p}{\tensor{\overset{\sim}{\varphi}}{_{\overline{\indiq[#1]{2}}}}}%
{A0p}{\tensor{\varphi}{_\perp}}%
{A0m}{\tensor[^{\text{P}}]{\varphi}{}}%
{A1p}{\tensor{\overset{\wedge}{\varphi}}{_{\perp\overline{\indiq[#1]{2}}}}}%
{A1m}{\tensor{\overset{\rightharpoonup}{\varphi}}{_{\overline{\indiq[#1]{1}}}}}%
{A2p}{\tensor{\overset{\sim}{\varphi}}{_{\perp\overline{\indiq[#1]{2}}}}}%
{A2m}{\tensor[^{\text{T}}]{\varphi}{_{\overline{\indiq[#1]{3}}}}}%
}[\packageError{cosmicclass}{Unidentified Critical Case: #1}{}]%
}

\newrobustcmd{\PiP}[2][placeholder]{%
\IfEqCase{#2}{%
{B0p}{\hat{\pi}}%
{B1p}{\tensor{\overset{\wedge}{\hat{\pi}}}{_{\overline{\indiq[#1]{2}}}}}%
{B1m}{\tensor{\hat{\pi}}{_{\perp\overline{\indiq[#1]{1}}}}}%
{B2p}{\tensor{\overset{\sim}{\hat{\pi}}}{_{\overline{\indiq[#1]{2}}}}}%
{A0p}{\tensor{\hat{\pi}}{_\perp}}%
{A0m}{\tensor[^{\text{P}}]{\hat{\pi}}{}}%
{A1p}{\tensor{\overset{\wedge}{\hat{\pi}}}{_{\perp\overline{\indiq[#1]{2}}}}}%
{A1m}{\tensor{\overset{\rightharpoonup}{\hat{\pi}}}{_{\overline{\indiq[#1]{1}}}}}%
{A2p}{\tensor{\overset{\sim}{\hat{\pi}}}{_{\perp\overline{\indiq[#1]{2}}}}}%
{A2m}{\tensor[^{\text{T}}]{\hat{\pi}}{_{\overline{\indiq[#1]{3}}}}}%
}[\packageError{cosmicclass}{Unidentified Critical Case: #1}{}]%
}

\newcommand{\APiP}{\pi}
\newcommand{\BPiP}{\pi}
%\newcommand{\APiP}{\tensor{\hat{\pi}}{_{ij\overline{k}}}}
%\newcommand{\BPiP}{\tensor{\hat{\pi}}{_{i\overline{k}}}}
\newcommand{\surface}{\partial}
 %this formats the Class X notation for the various cosmologies

\usepackage{hyperref}
\hypersetup{
     colorlinks = true,
     linkcolor = Blue,
     citecolor = Blue,
     filecolor = Blue,
     urlcolor = Blue 
     }
\usepackage[capitalize]{cleveref} %always load this last in preamble

\begin{document}
\setpagewiselikenumbers
%\modulolinenumbers[5]

\title{Nonlinear Hamiltonian analysis of the novel Poincar{\'e} gauge theories}

\author{W.E.V. Barker}
\email{wb263@cam.ac.uk}
\affiliation{Astrophysics Group, Cavendish Laboratory, JJ Thomson Avenue, Cambridge CB3 0HE, UK}
\affiliation{Kavli Institute for Cosmology, Madingley Road, Cambridge CB3 0HA, UK}
\author{A.N. Lasenby}
\email{a.n.lasenby@mrao.cam.ac.uk}
\affiliation{Astrophysics Group, Cavendish Laboratory, JJ Thomson Avenue, Cambridge CB3 0HE, UK}
\affiliation{Kavli Institute for Cosmology, Madingley Road, Cambridge CB3 0HA, UK}
\author{M.P. Hobson}
\email{mph@mrao.cam.ac.uk}
\affiliation{Astrophysics Group, Cavendish Laboratory, JJ Thomson Avenue, Cambridge CB3 0HE, UK}
\author{W.J. Handley}
\email{wh260@cam.ac.uk}
\affiliation{Astrophysics Group, Cavendish Laboratory, JJ Thomson Avenue, Cambridge CB3 0HE, UK}
\affiliation{Kavli Institute for Cosmology, Madingley Road, Cambridge CB3 0HA, UK}


%\date{}

\begin{abstract}

\end{abstract}

\pacs{04.50.Kd, 04.60.-m, 04.20.Fy, 98.80.-k, 90.80.Es}

\maketitle

\section{Introduction}
Break quadratic action down using irreducible projection operators (IPOs) with respect to $\mathrm{SO}(1,3)$ -- six for curvature and three for torsion. We write this as
\begin{equation}
\begin{aligned}
  L_{\text{T}}=&\ \sum_{I=1}^{3}{m_\text{p}}^2\tensor{\hat{\beta}}{_I}\tensor{\mathcal{  T}}{^{i}_{jk}}\tensor[^I]{\mathcal{  P}}{_{i}^{jk}_{l}^{nm}}\tensor{\mathcal{  T}}{^{l}_{nm}}\\
  &+\sum_{I=1}^{6}\tensor{\hat{\alpha}}{_I}\tensor{\mathcal{  R}}{^{ij}_{kl}}\tensor[^I]{\mathcal{  P}}{_{ij}^{kl}_{nm}^{pq}}\tensor{\mathcal{  R}}{^{nm}_{pq}}+L_{\text{m}}
  \label{<+label+>}
\end{aligned}
\end{equation}
The \emph{canonical} momenta are 
\begin{equation}
  \tensor{\pi}{_i^\mu}\equiv\frac{\partial bL}{\partial\partial_0\tensor{b}{^i_{\mu}}}, \quad \tensor{\pi}{_{ij}^{\mu}}\equiv\frac{\partial bL}{\partial\partial_0\tensor{A}{^{ij}_{\mu}}}
  \label{<+label+>}
\end{equation}
from which we define \emph{parallel} momenta $\tensor{\hat{\pi}}{_{i}^{\overline{k}}}\equiv\tensor{\pi}{_{i}^\alpha}\tensor{b}{^{k}_\alpha}$ and $\tensor{\hat{\pi}}{_{ij}^{\overline{k}}}\equiv\tensor{\pi}{_{ij}^\alpha}\tensor{b}{^{k}_\alpha}$, which are
\begin{subequations}
  \begin{align}
    J^{-1}\tensor{\hat{\pi}}{_{i}^{\overline{k}}}=&\ \frac{\partial L}{\partial \tensor{\mathcal{  T}}{^i_{\perp\overline{k}}}}=\sum_{I=1}^{3}4{m_\text{p}}^2\tensor{\hat{\beta}}{_{I}}\tensor[^I]{\mathcal{  P}}{_i^{\perp\overline{k}}_n^{mo}}\tensor{\mathcal{  T}}{^n_{mo}},\\
    J^{-1}\tensor{\hat{\pi}}{_{ij}^{\overline{k}}}=&\ \frac{\partial L}{\partial \tensor{\mathcal{  R}}{^{ij}_{\perp\overline{k}}}}=\sum_{I=1}^{6}8\tensor{\hat{\alpha}}{_{I}}\tensor[^I]{\mathcal{  P}}{_{ij}^{\perp\overline{k}}_{mn}^{pq}}\tensor{\mathcal{  R}}{^{mn}_{pq}}.
    \label{<+label+>}
  \end{align}
  \label{<+label+>}
\end{subequations}
Next we project out $\mathrm{O}(3)$ irreps using $\tensor{n}{_i}$ to break Lorentz symmetry (not really broken because of lapse and shift). 
\begin{subequations}
\begin{align}
  \pic{B0p}&\equiv J^{-1}\PiP{B0p}=,\\
  \pic{B1p}&\equiv J^{-1}\PiP{B1p}=,\\
  \pic{B1m}&\equiv J^{-1}\PiP{B1m}=,\\
  \pic{B2p}&\equiv J^{-1}\PiP{B2p}=,\\
  \pic{A0p}&\equiv J^{-1}\PiP{A0p}=,\\
  \pic{A0m}&\equiv J^{-1}\PiP{A0m}=,\\
  \pic{A1p}&\equiv J^{-1}\PiP{A1p}=,\\
  \pic{A1m}&\equiv J^{-1}\PiP{A1m}=,\\
  \pic{A2p}&\equiv J^{-1}\PiP{A2p}=,\\
  \pic{A2m}&\equiv J^{-1}\PiP{A2m}=.
\end{align}
\end{subequations}
Next we evaluate Poisson brackets between these. The following very useful identities for this process in ADM are not given in the literature: 
\begin{equation}
  \begin{gathered}
    \frac{\partial\tensor{n}{_l}}{\partial\tensor{b}{^k_\mu}}=-\tensor{n}{_k}\tensor{h}{_{\overline{l}}^\mu}, \quad \frac{\partial\tensor{h}{_l^\nu}}{\partial\tensor{b}{^k_\mu}}=-\tensor{h}{_k^\nu}\tensor{h}{_l^\nu},\\
    \frac{\partial b}{\partial\tensor{b}{^k_\nu}}=b\tensor{h}{_k^\nu}, \quad \frac{\partial J}{\partial\tensor{b}{^k_\nu}}=J\tensor{h}{_{\overline{k}}^\nu}.
  \end{gathered}
  \label{<+label+>}
\end{equation}

The super-Hamiltonian is
\begin{equation}
  \mathcal{  H}_{\perp}\equiv\tensor{\hat{\pi}}{_i^{\overline{k}}}\tensor{\mathcal{  T}}{^i_{\perp\overline{k}}}+\frac{1}{2}\tensor{\hat{\pi}}{_{ij}^{\overline{k}}}\tensor{\mathcal{  R}}{^{ij}_{\perp\overline{k}}}-JL-\tensor{n}{^k}\tensor{D}{_\alpha}\tensor{\pi}{_k^\alpha}
  \label{<+label+>}
\end{equation}
from which we eventually get
\begin{equation}
\begin{aligned}
  \mathcal{  H}_{\perp}=&\ \sum_{I=1}^{3}{m_\text{p}}^2J\tensor{\hat{\beta}}{_{I}}\Big[4\tensor{\mathcal{  T}}{^i_{\perp\overline{k}}}\tensor[^I]{\mathcal{  P}}{_i^{\perp\overline{k}}_j^{\perp\overline{l}}}\tensor{\mathcal{  T}}{^j_{\perp\overline{l}}}\\
  &-\tensor{\mathcal{  T}}{^i_{\overline{mk}}}\tensor[^I]{\mathcal{  P}}{_i^{\overline{mk}}_j^{\overline{nl}}}\tensor{\mathcal{  T}}{^j_{\overline{nl}}}\Big]\\
  +&\ \sum_{I=1}^{6}J\tensor{\hat{\alpha}}{_{I}}\Big[4\tensor{\mathcal{  R}}{^{ip}_{\perp\overline{k}}}\tensor[^I]{\mathcal{  P}}{_{ip}^{\perp\overline{k}}_{jq}^{\perp\overline{l}}}\tensor{\mathcal{  R}}{^{jq}_{\perp\overline{l}}}\\
  &-\tensor{\mathcal{  R}}{^{ip}_{\overline{mk}}}\tensor[^I]{\mathcal{  P}}{_{ip}^{\overline{mk}}_{jq}^{\overline{nl}}}\tensor{\mathcal{  R}}{^{jq}_{\overline{nl}}}\Big]-\tensor{n}{^k}\tensor{D}{_\alpha}\tensor{\pi}{_k^\alpha}\\
  \label{<+label+>}
\end{aligned}
\end{equation}
-- thus easy to see that super-Hamiltonian decomposes into quadratic canonical terms and terms which can be made canonical by substituting with any nonvanishing constraint functions.











\section{Massive theories}

The simplest theories (constraints are only functions of momenta) turn out to all be massive-only in the weak regime. Obviously this is quite useless, but we can use it as a staging area for the supposedly viable theories. We construct the Poisson matrices of primary if-constraints (PICs) to be strictly on-shell. The entries are purely schematic, and represent some usually long expressions which currently exist only on paper. As expected, all nonzero Poisson brackets for the selected cases are linear in the momenta. The labels around the edge give the $\mathrm{O}(3)$ irreps of the constraint fields, with multiplicities. For ease, the matrices are divided into quadrants (should more than one quadrant exist) for translational, rotational and mixed brackets. The mixed brackets are useful for chain termination.

Case 20:
\begin{equation}
  \input{Case-20}
  \label{<+label+>}
\end{equation}
Case 24:
\begin{equation}
  \input{Case-24}
  \label{<+label+>}
\end{equation}
Case 25:
\begin{equation}
  \input{Case-25}
  \label{<+label+>}
\end{equation}
Case 26:
\begin{equation}
  \input{Case-26}
  \label{<+label+>}
\end{equation}
Case 28:
\begin{equation}
  \input{Case-28}
  \label{<+label+>}
\end{equation}
Case 32:
\begin{equation}
  \input{Case-32}
  \label{<+label+>}
\end{equation}

\section{Massless theories}

From general considerations, there are very serious problems with all the novel theories which have massless modes.

Case 3:
\begin{equation}
  \input{Case-3}
  \label{<+label+>}
\end{equation}

Case 17:
\begin{equation}
  \input{Case-17}
  \label{<+label+>}
\end{equation}
Currently evaluating
\begin{equation}
  \input{test.txt}
  \label{<+label+>}
\end{equation}

\begin{acknowledgments}


\end{acknowledgments}

\bibliographystyle{apsrev4-1}
\bibliography{bibliography}

\end{document}
